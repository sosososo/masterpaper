\chapter{基于深度学习的缺陷检测}


本章将详细介绍如何将法线图应用到深度学习中,对零件进行缺陷检测。
首先我们会介绍一些深度学习的基本概念,
包括卷积层、激活函数、池化层、全连接层等,
接着介绍本文使用的神经网络模型的网络结构,
最后介绍如何对模型进行训练和测试,并给出实验结果和分析。

\section{基本概念}

本文使用卷积神经网络($Convolutional Neural Networks,\mbox{简称}~CNN$)结构进行缺陷检测。
卷积神经网络是一种层次模型,
它的输入是一张图像,
接着通过卷积($convolution$)操作、
池化\cite{chechik1998synaptic}($pooling$)操作和非线性激活函数($non-linear~activation~function$)映射等一系列操作的堆叠,
逐渐从原始图像提取出高维特征,
最后通过全连接层和输出层连接,
将目标任务(分类、回归等)形式化为目标函数,
并使用损失函数($loss function$)
计算预测值和真实值之间的损失($loss$),
使用反向传播算法\cite{周志华2016机器学习}($back-propagation algorithm$)将损失从后向前传播,更新参数,
如此反复直到模型收敛。

好的卷积神经网络模型得益于良好的网络结构设计,
而一个复杂精妙的网络结构往往由诸多基本结构组成,
本节首先介绍构成神经网络的一些基本结构,
这些结构将在接下来的网络结构以及实验等部分用到,
接着介绍网络模型的损失函数和优化方法。

\subsection{卷积层}

卷积是卷积神经网络中的基础操作。所谓卷积,就是使用一个固定大小的卷积核,通过滑动窗口的形式和图像对应区域做內积(逐个元素相乘再求和)操作,此时卷积核为一个固定大小的矩阵,这些矩阵的值即为该卷积核的权值。一般我们使用正方形的卷积核,并且其大小取奇数,常用的有$1\times 1$、$3\times 3$、$5\times 5$的卷积核,卷积核的大小即对应了滑动窗口的大小。
每完成一次卷积操作,卷积核移动一个位置,移动的大小记为卷积步长($stride$)。
因为图像是二维的,为了卷积整幅图像,需要卷积核向两个方向移动,因此卷积步长也是二维的,一般取$1\times 1$。

不难看出,卷积是一种局部操作,通过一定大小的卷积核作用于图像的局部区域,从而得到图像的局部信息,通过层层卷积,高层的卷积核可以扫过的信息会覆盖原始图片的更大区域,提取图像更高层的信息。需要注意的是,图片的底层特征往往与其在图片中的位置无关,因此我们在对图像做一次卷积时使用权值相同的卷积核,这被称为“权值共享”。权值共享大大减少了卷积层的参数,起到了防止过拟合的作用。

\subsection{激活函数}

激活函数层又称为非线性映射层,其引入正是为了增加整个网络的表达能力(非线性)。如果没有激活函数层,那么整个网络仍可以看作是若干线性操作的堆叠,只能起到线性映射的作用。直观的看,激活函数模拟了神经元的特点:接收一组输入信号并产生输出。
激活函数的输入往往是卷积层卷积之后得到的值,因此通常激活函数层紧跟着卷积层,被合并为一层。

从定义来看,几乎所有的连续可导函数都可以作为激活函数,但是目前常见的多是分段性和具有指数形状的非线性函数,
本小节将介绍几种最基础的激活函数,
包括$Sigmoid$\cite{魏秀参解析卷积神经网络}函数、$Tanh$\cite{魏秀参解析卷积神经网络}函数和$ReLU$\cite{魏秀参解析卷积神经网络}函数。

\subsubsection{$Sigmoid$ 函数}

$Sigmoid$是使用范围最广的一类激活函数,
它是具有指数函数形状的激活函数。该函数的定义如公式\eqref{eq:sigmoid}所示。
\begin{equation}
\centering
f(x)=\frac{1}{1+e^{-x}}
\label{eq:sigmoid}
\end{equation}
其函数形状如图\ref{fig:Sigmoid}所示,从图中可以明显看出,经过$Sigmoid$型函数作用后,输出响应值的值域被压缩到$(0,1)$之间。
$Sigmoid$ 有其有利的一面,它在物理意义上最为接近生物神经元,并且$(0,1)$的输出范围可以理解为一个概率分布。
但是不难看出,
对于大于5(或小于-5)的值,无论多大(或多小)都会被压缩为1(或0),此部分的梯度会接近0,
从而导致在误差反向传播过程中该区域的误差很难传播,进而导致模型无法收敛,这一现象被称为“梯度消失”。
梯度消失可以被其它优化方法缓解,如Batch Normalization\cite{ioffe2015batch}。
\begin{figure}[htbp]
\centering
\includegraphics[width=0.6\linewidth]{figures/sigmoid.png} 
\caption{$Sigmoid$激活函数示意图}
\label{fig:Sigmoid}
\end{figure} 

\subsubsection{$Tanh$ 函数}

$Tanh$激活函数又被称为双正切函数,
其形状与$Sigmoid$函数类似,都能将输出压缩到$(0,1)$,
但是该函数输出的均值比$sigmoid$函数更接近0,在训练时收敛速度更快。
$Tanh$函数形式如公式\eqref{eq:tanh}所示。
\begin{equation}
tanh⁡(x)=\frac{1-e^{-2x}}{1+e^{-2x}}
\label{eq:tanh}
\end{equation}
$Tanh$函数形状如图\ref{fig:tanh}所示,不难看出,
$tanh$函数在$x$比较大或者比较小的时候仍会出现梯度消失的现象。
\begin{figure}[htbp]
\centering
\includegraphics[width=0.6\linewidth]{figures/tanh.png}
\caption{$tanh$激活函数示意图}
\label{fig:tanh}
\end{figure}

\subsubsection{$ReLU$ 函数}

$ReLU$\cite{nair2010rectified}是由Nair和Hinton在2010年提出的,
现已成为深度学习中最常用的激活函数之一。$ReLU$函数实际上是一个分段函数,其函数形式如公式\eqref{eq:relu}所示。
\begin{equation}
\centering
ReLU(x)=\max\{0,x\}
\label{eq:relu}
\end{equation}
与前两个激活函数相比,$ReLU$函数的梯度在$x\geq 0$时为1,反之为0,如图\ref{fig:relu}所示,
在$x\geq 0$的部分,该函数完全消除了梯度消失的情况,并且$ReLU$的计算复杂度更低,
在实际应用中能够更快的收敛。
$ReLU$的主要缺陷是在$x<0$时,
梯度为0,函数对这部分的卷积结果无法响应,
因此输入一旦变为负值将再无法影响网络的训练。
这一缺陷可以通过$Parametric~Rectified~Linear~Unit(PReLU)$\cite{he2015delving}来改善。
\begin{figure}[htbp]
\centering
\includegraphics[width=0.6\linewidth]{figures/relu.png}
\caption{$ReLU$激活函数示意图}
\label{fig:relu}
\end{figure}

\subsection{池化层}

池化层一般采用平均池化($average-pooling$)或最大池化($max-pooling$)。
类似于卷积操作,池化层选择一个滑动窗口,取窗口中像素值最大值(最大池化层)或者平均值(平均池化层)作为池化操作的输出,可以对特征进行降维。从操作的角度来讲,池化层可以看做是一个用$p-$范数($p-norm$)作为非线性映射的“卷积”操作,当$p$趋近于正无穷时就是最大池化层。
与卷积层不同的是,池化层不需要设定参数,
使用时只需要指定池化层类型、池化操作的步长和核的大小即可。
从图像处理的角度来看,池化层可以被视为一种“降采样”,
一般卷积操作之后得到的特征可能会包含较多冗余信息,池化层的引入可以对特征进行降维和抽象,在一定程度上预防过拟合。
虽然池化层不是卷积神经网络中必须的,但是由于其良好的性质,往往会在卷积层之后使用池化层对特征降维。

\subsection{全连接层}

全连接层往往在整个卷积神经网络的最后一层或几层,起到“分类器”的作用。在这一层,每一个神经节点都会与上一层中的所有节点相连,因此全连接层往往含有较多参数。
如果说卷积层、激活函数和池化层是用来将原始数据映射到高维特征空间中的话,
那么全连接层就起到将特征映射到样本的标记空间中的作用。
实际应用中全连接层可以通过卷积操作实现,
当其前一层为全连接层时,可以通过$1\times 1$的卷积核实现,
当其前一层为卷积层时,可以通过与前一层特征维度大小一致的卷积核做全局卷积实现。

\subsection{损失函数}
\label{subseciont:sunshihanshu}

损失函数又被称为目标函数,被用来衡量预测值和真实样本标记之间的误差。在卷积神经网络中,
分类问题常用交叉熵损失函数作为目标函数,回归问题常用$l_2$损失函数作为目标函数。
\begin{equation}
\centering
L_{loss}=-\frac{1}{N}\sum_{i=1}^N(y_i\log{\hat{y_i}+(1-y_i)\log{(1-\hat{y_i})}})
\label{eq:lloss}
\end{equation}
本文将缺陷检测定义为一个二分类问题,
使用交叉熵损失函数$L_{loss}$作为目标函数,
其公式定义如\eqref{eq:lloss}所示,
其中N表示样本的数量,
$L_{loss}$表示整体损失。
$y_i$表示样本标签,当样本为缺陷样本时,
$y_i$值为1,
当样本为正常样本时,$y_i$值为0,
$\hat{y_i}$表示模型的输出,
它可以被看作该样本是缺陷样本的概率。
最终模型会输出样本为每个类的概率,
我们取概率最大的类作为最终预测的类。

\subsection{优化方法}

有了目标函数之后就可以针对目标函数优化网络参数,深度卷积神经网络通常采用随机梯度下降类的优化算法对模型参数优化。本节将介绍几种最常用的优化算法。

\subsubsection{随机梯度下降}

随机梯度下降算法($Stochastic~Gradient~Descent,\mbox{简称}~\mbox{SGD}$)是神经网络中最基础的优化算法,它根据误差的一阶梯度信息对参数调整,参数的更新策略可以用公式\eqref{eq:sgd}表示,
\begin{equation}
\centering
w=w-\eta \cdot dw
\label{eq:sgd}
\end{equation}
其中,$dw$表示误差对参数w的导数即梯度,它依赖于当前数据在目标函数上的误差,
我们利用误差的反向传播算法对其求解;
$\eta$表示学习速率,
是SGD算法中唯一的超参数,表示当前梯度值对网络参数更新的影响程度。
SGD算法收敛效果稳定,但是收敛速度较慢,并且,权值容易被困在鞍点,即$dw=0$的点,导致模型无法完全优化。SGD中学习率的设定也是一个问题,在选择学习率的时候,过大的学习率可能导致模型在训练阶段后期来回震荡,无法收敛,过小的学习率又会影响收敛速度。

\subsubsection{基于动量的随机梯度下降法}

受物理学研究的启发,
发展出了基于动量\cite{qian1999momentum}($momentum$)的随机梯度下降算法,
该算法通过前几轮训练积累的“动量”信息辅助参数更新,更新策略用公式\eqref{eq:momentum}表示,
\begin{equation}
\begin{aligned}
\centering
v & =\mu \cdot v-\eta \cdot dw \\
w & =w+v
\end{aligned}
\label{eq:momentum}
\end{equation}
公式中,$\mu$为动量因子,表示动量对整体梯度更新的影响程度,
一般$\mu$取0.9;
$v$表示动量,在梯度方向相同的方向逐渐增大,
在梯度方向不同的方向逐渐变小;
$\eta$为学习率,
表示梯度$dw$对动量的影响。
基于动量的随机梯度下降法可以抑制SDG中会出现的震荡现象,
还可以帮助跳出鞍点,找到参数的更优解。

\subsubsection{RMSProp算法}

RMSProp($root~mean~square~prop$)算法\cite{hinton2012neural}可以针对学习率做动态的调整。这一算法的更新策略如\eqref{eq:rmsprop}所示,
\begin{equation}
\begin{aligned}
\centering
Sdw & =\beta \cdot Sdw-(1-\beta )dw^2 \\
w & =w- \alpha \frac{dw}{\sqrt{Sdw}+\varepsilon }
\end{aligned}
\label{eq:rmsprop}
\end{equation}
公式中加入了w的二阶导数对$Sdw$进行更新,在更新$w$的时候使用用$w$的梯度除以$Sdw$的平方根作为学习对象,
这使得对于不同的w可以有不同的学习率。其中$\varepsilon$只是为了防止分母变为0,本身对算法的意义不大,一般置为${10}^{-6}$,$\beta$为衰减因子,用于消除算法对全局学习率$\alpha$的依赖,
较大的$\beta$会促进网络更新,较小的$\beta$会抑制网络更新,一般可以取0.9,$\alpha$可以取1。

\subsubsection{Adam算法}

Adam\cite{kingma2014adam}算法用梯度的一阶矩估计和二阶矩估计动态调整每个参数的学习率,并且经过偏置校正后,每一次迭代学习率都有一个确定的范围,可以使得参数更新更加平稳。算法使用公式\eqref{eq:adam}调整参数$w$,调整时首先初始化$Vdw=0$,$Sdw=0$,
接着不断迭代更新参数,直到模型收敛。
\begin{equation}
\begin{aligned}
\centering
&Vdw  =\beta_1 \cdot Vdw-(1-\beta_1 )dw \\
&Sdw  =\beta_2 \cdot Sdw-(1-\beta_2 )dw^2 \\
&Vdw^{corrected}  =\frac{Vdw}{1-\beta_1^t}\\
&Sdw^{corrected}  =\frac{Sdw}{1-\beta_2^t}\\
&w  =w- \alpha \frac{Vdw^{corrected}}{\sqrt{Sdw^{corrected}}+\varepsilon }
\end{aligned}
\label{eq:adam}
\end{equation}
可以看出,Adam算法也需要指定参数,其中$\varepsilon$是为了防止分母为0,
使用${10}^{-6}$即可;$\beta_1$为第一矩参数,可以使用0.9;$\beta_2$为第二矩参数,
可以使用0.999。
该算法既考虑了动量又可以针对不同的参数调整学习率,常常具有较快的收敛速度和较好的效果。本文使用Adam算法对模型进行优化。

\section{深度学习模型}

本文使用VGG-11\cite{simonyan2014very}模型,
并对其进行了一定修改。
模型结构如图\ref{fig:VGG}所示,
图中用不同颜色的矩形标示出了不同种类的网络层。
第一层为输入层,输入图片的大小为$224\times224\times3$;
黑色矩形表示卷积层,
卷积核大小为$3\times3$,
步长为2。
论文中VGG11的卷积层后面紧跟$ReLU$激活层,
本文在卷积层之后、$ReLU$层之前加入了批归一化($Batch~Normalization$)层,
批归一化层由Sergey loffe和Christian Szegedy等人提出,
可以将数据归一化至均值为0、方差为1的分布。
这种方法有两个好处:首先它把输入的均值、方差规范化,保证了每一层的输入分布不发生改变,能够加快模型收敛;
其次它能起到轻微正则化的作用,预防过拟合。
红色矩形表示最大池化层,池化窗口大小为$2\times2$,步长为2。
浅蓝色矩形表示全连接层和紧随其后的$ReLU$激活层。
深蓝色矩形表示全连接层。
褐色矩形表示$Softmax$层,用于分类。

\begin{figure}[htbp]
\centering
\includegraphics[width=1.0\linewidth]{figures/VGG.png}
\caption{VG-11模型结构示意图}
\label{fig:VGG}
\end{figure}

在训练阶段,我们使用$dropout$方法来预防过拟合。
$Dropout$在向前传播时对神经元以一定的概率随机失活,这种失活只是暂时的,
每一轮训练时,都会以固定的概率重新失活神经元。
本文中$Dropout$失活概率取0.5。


\section{数据预处理}

同基于传统方法的缺陷检测一样,在使用深度学习进行缺陷检测时,数据的预处理也是非常重要的一环。
本章中的数据预处理方法和第\ref{chapter:chuantongfangfa}章类似,
我们首先提取零件的主表面,
接着用数据增强的方法扩大数据规模。

在提取零件主表面阶段,我们使用的方法没有任何变化,此处不再赘述。
在数据增强阶段,我们同样首先对数据做分割,
接着使用镜像和旋转增加样本数量,
但是VGG-11的输入是$224\times 224$,
同\ref{subsubssection:chuantongfenge}节中描述的图像分割方法不同,
本节中滑动窗口的大小为$40\times 40$,
接着我们使用双三次插值法将其放大为$224\times 224$。

\section{模型训练}

在训练阶段阶段,
首先,
我们用\ref{subsection:jilianjianceqi}中提到的方法
将数据集划分为不同的子数据集,
对每个子数据集训练一个VGG-11模型,
最后使用级联的方式将其组合成一个级联检测器。
在模型训练时,
首先初始化模型,
然后优化参数,
直到模型收敛。
初始化模型时,
我们选择在ImageNet上预训练好的参数权重初始化模型。
在优化参数时,
优化的目标函数为\ref{subseciont:sunshihanshu}节定义的损失函数,
在每一轮迭代时,使用$mini-batch$对模型训练,$mini-batch$的大小设置为48,
并使用Adam算法对模型进行优化。
需要注意的是,我们并非选择收敛后的模型作为最终模型,
而是边训练边对模型进行评估,选择测试集上效果最好的模型作为最终模型。


\section{实验结果和分析}

本节将展示使用本章算法进行缺陷检测得到的结果,并将其与传统算法进行对比。
实验使用的深度学习框架是Pytorch,
该框架是深度学习中常用的框架之一,它能够支持多种平台,现已被广泛应用于学术界和工业界中。

同基于传统方法的缺陷检测算法一样,
我们使用
召回率($REC$)、
查准率($PRE$)、
漏检率( $MDR$)、
误检率($FDR$)
作为模型评价标准,
,其结果如表\ref{tab:shenduxuexijieguo}所示。我们发现,本章算法的实验结果并不比使用HOG、Gredient和RGB联合特征训练出来的传统模型的结果好,我们认为最主要的原因有两点:首先在VGG-11中,输入数据的大小为$224\times 224$,虽然这是原始数据放大之后的结果,但由于缺陷目标往往较小,在这样的输入中,缺陷目标所占比例很小,而模型网络又比较深,使得缺陷区域的特征在最后的特征层中占比更小;其次,受限于缺陷零件数据量,模型容易过拟合。
\begin{table}
\centering
\begin{tabular}{cccccp{38mm}}
\toprule
\textbf{输入类型} & \textbf{REC} & \textbf{PRE} & \textbf{MDR} & \textbf{FDR}\\
\midrule
\mbox{HOG+Gredient+RGB} & 0.9915 & 0.8158 & 0.0085 & 0.0400\\
\mbox{VGG-11} & 0.9643 & 0.7297 & 0.0356 & 0.1786\\
\bottomrule
\end{tabular}
\caption{深度学习和传统方法检测结果表}
\label{tab:shenduxuexijieguo}
\end{table}

在检测速度方面,深度学习由于其复杂的网络结构和较多的网络参数,速度并不快,但由于深度学习和传统方法的输入大小不同,此处并不对其进行比较。
