%%%%%%%%%%%%%%%%%%%%%%%%%%%%%%%%%%%%%%%%%%%%%%%%%%%%%%%%%%%%%%%%%%%%%%%%%%%%%%%
% 致谢,应放在《结论》之后
\begin{acknowledgement}

从选择了读研,到如今论文完成,时光转眼走过三年。这三年来,有数不清的欢笑与汗水,感动与收获,有太多的人需要我去感谢!

首先,最需要感谢的是我的导师郭延文教授。在论文的编写中,郭老师不仅在选题上给予了我悉心的指导,更是在实验和撰写过程中给出了独特的意见和帮助。作为一名教授,郭博士在学术上拥有极高造诣,是我学术道路上的引路人和指导者,作为一名老师,郭老师以身作则,以严谨的态度和勤奋的精神树立了了我工作、学习和生活上的典范。从老师那里,我得到不仅仅是宝贵的知识财富,更是受益一生的做人道理。

我还要感谢周文喆师兄和吕高建师姐,我们 一起设计了获取物体表面材质的设备,并实现了相关算法,这一过程工程量庞大,仅凭我一人之力可能难以完成,是和他们的通力合作,才完成了这一工作任务,在和他们的合作中我也学到了很多的知识,获得了快速的成长。我要感谢张扬师兄,在将法线信息应用到目标检测的过程中,他以渊博的学术见解和深厚的编程功底为我提供了极大的帮助。感谢同门的于霄和黄凯同学,我们同一级入学,一起学习和生活了诸多时间,这些时间带给我非常多的帮助、快乐和感动,并且在最后的论文撰写和修改阶段,我们互帮互助,顺利完成了各自的论文。感谢你们。

感谢刘明明、朱捷、潘飞、强玉庭、张宏杰、贺敬武五位博士师兄在三年来的帮助和指导,感谢马晗师兄的热心帮助,感谢张可心师妹、张云峰师弟、陈钊民师弟、韩旭师妹,感谢张慧明师弟、陈玉念师妹、罗曼琳师妹、徐春雷师妹,感谢能够相遇在实验室的所有人,感谢你们。

感谢我的舍友孙佳俊、宋仁杰,三年的舍友情弥足珍贵,感谢在南京大学遇到的熊宇、王铖燕等所有的朋友们,是和你们的相遇填满了我研究生生涯的宝贵时光,感谢你们。

最后,我要感谢我的家人,感谢家人给予我的关心与照顾,是他们的支持和陪伴,让我有勇气和力量一路走下去,感谢你们,我爱你们。

感谢!




\end{acknowledgement}