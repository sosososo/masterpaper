\begin{abstract}

近些年来,计算机视觉和机器学习的不断发展取得了瞩目的成功,不仅在学术界中被诸多学者研究,在工业界中也被大量应用。
其中,目标检测是非常基础和经典的课题,而缺陷检测则是目标检测极具工业价值的应用领域,
传统的目标检测算法首先提取图像中的特征,然后将特征作为分类器的输入对图像分类。其缺点是对特征选择要求严格,并且分类模型泛化能力较弱,容易受成像效果、零件纹理、光照条件和噪声等因素的影响。
随着深度学习的蓬勃发展,越来越多的深度学习算法被应用到目标检测中,在某些领域尤其是竞赛中取得了理想的效果。但是这些方法一般存在两个缺点,首先模型训练需要大量数据;其次模型对小目标效果不理想,而在零件缺陷检测中,缺陷往往较小。

针对传统算法和深度学习算法各自的缺点,本文首先设计了提取零件法线图的方法,接着将其作为传统缺陷检测算法的输入,取得了理想的效果。我们还将法线图应用到深度学习算法中,对比和分析了二者的结果。
本文的主要工作和创新有:
\begin{enumerate}

\item 首先,本文提出了一种零件表面法线提取算法。该算法使用不同方向光源下拍摄得到的零件照片作为输入,基于光线的反射原理计算零件表面法线。
首先,我们设计了一套硬件设备,该设备包括遮光模块、平台模块、灯光模块、拍照模块和控制模块五大模块,这些模块将相机、步进电机、LED灯带、滑轨、偏光镜和滤光膜等不同硬件组合成一个整体,
可以捕获不同方向光源下的零件照片。
接着我们将设备得到的零件照片作为法线计算算法的输入,提取零件法线。
为了优化照片的质量,提高零件法线的精度,
在实施法线计算算法之前,
本文首先对相机进行校正,
并对方向光源产生的光线损失进行光线补偿。
校正算法包括白平衡校正、色彩校正、畸变校正。
补偿算法首先预存储一套补偿数据,接着利用它对新的照片进行补偿。
值得一提的是,我们的法线提取算法不仅可以用来获取零件法线,还可以用来获取其他物体的法线。

\item 其次,本文将法线图作为缺陷检测算法的输入,应用到零件缺陷检测算法中。
本文首先使用传统检测算法进行检测,算法包括数据预处理、特征提取、设计和训练分类器以及用分类器进行检测等步骤。
在数据预处理阶段,
我们首先使用零件主表面提取算法滤除照片中的无关背景,
接着使用数据增强的方法扩大数据规模。
在特征提取阶段,我们分别提取了Haar-like特征、图像梯度特征、LBP特征、GLCM特征、HOG特征等不同的图像特征,并分析了不同特征的分类效果,根据分类效果对特征进行组合,从而得到一组最优特征组合。
在分类器分类阶段,考虑到由于缺陷零件较少造成的样本不平衡的问题,我们使用随机抽样的方法将数据集划分为不同的子数据集,每个子数据集中样本是平衡的。
在训练阶段,我们使用支持向量机作为基分类器,对于每个子数据集,我们训练多个不同的基分类器,并利用Adaboost的方法将其组合成一个强分类器,
最终每个子数据集得到一个强分类器,
在检测阶段,
我们将这些强分类器组合成级联检测器,进行缺陷检测。

\item 最后,本文还将法线图作为深度学习模型的输入,训练了一个深度学习模型。
本文首先使用滑动窗口的方式将法线图分割成不同的窗口,
接着使用分类模型对这些窗口做分类。
在准备数据阶段,我们使用与传统算法中相似的预处理方法对数据进行处理。
在模型设计阶段,我们针对零件缺陷检测调整和优化了模型。
在模型训练和检测阶段,我们同样将数据集划分为不同的子数据集,
每个子数据集训练一个模型,
并利用级联的方式将它们组成一个级联检测器。
最后本文还将深度学习模型的检测结果与传统检测方法的结果进行了对比,分析了二者各自的优缺点及原因。

\end{enumerate}
% 中文关键词。关键词之间用中文全角分号隔开,末尾无标点符号。
\keywords{法线图;缺陷检测;支持向量机;深度学习}
\end{abstract}
